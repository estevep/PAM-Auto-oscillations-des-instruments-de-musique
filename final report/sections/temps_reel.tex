\subsection{État de l'art}

Plusieurs outils de programmation d'applications musicales interactives sont disponibles et ont été utilisés avec succès pour de précédents travaux. 
Nous pourrons citer 
\href{https://cycling74.com/products/max}{Max},
\href{https://faust.grame.fr/}{Faust}
et 
\href{https://puredata.info}{Pure Data}
comme principaux choix pour réduire la complexité des implémentations et permettant de travailler uniquement sur les algorithmes de traitement et les interfaces utilisateur. 
Pour un embarquement dans un microcontroleur et gagner l'autonomie de l'instruments, plusieurs solutions existent offrant la possibilité de travailler avec les outils susnommés comme le projet \href{https://electro-smith.com/}{Daisy}.


\subsection{Implémentation et interface temps réel}

Pour cette étude a été choisi de réaliser l'application à l'aide de Max-MSP, offrant un environnement de programmation visuelle orienté vers les traitements numériques audio en temps réel et l'intéractivité avec l'utilisateur. 
Pour implémenter la méthode de Runge-Kutta \ref{sec:result_modal} à l'ordre quatre, il est nécessaire de traiter chaque échantillon du signal indépendemment.
Pour cela, Max propose l'environnement Gen qui, en plus de répondre à cette contrainte, offre une interface textuelle simplifiant l'implémentation.
Enfin, pour l'interface, l'intégration de Javascript dans le \textit{patch} étend les possibilités d'interaction avec l'utilisateur.
Il sera utilisé pour contraindre les paramètres de jeu dans les intervalles de validité de certains critères en fonction des choix de l'utilisateur.
Deux applications on été réalisées basées sur les deux modèles de synthèse abordés dans cette étude.


% @TODO Mettre éventuellement un lien vers un drive avec des sons ou des vidéos


\begin{comment}
    brouillon d'abstract
\end{comment}