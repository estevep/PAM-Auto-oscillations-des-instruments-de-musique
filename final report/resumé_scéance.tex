- fonction de réflection plus réaliste que dirac pour modèle de guide d'onde 
- ligne à retard et pas ligne à délais 
- couplage des modes : dans u c'est la somme de la pression totale et pas somme des p_n sommés à la fin. 
- somme des pressions modales pour chaque mode et pas de la pressions totale (regarder s'il y a de l'énergie dedans) 
- figure * pour prendre toute la page dans un double colonne
- intro et état de l'art peut fusionner 
- dans conclusion : qu'est ce qu'on a appris et en quoi ça couvre toutes les disciplines 
- section 4 : split en 4 partie qui incluent les méthodes et résultats de chaque parties. 
-doit être clair sur quand c'est résultats et quand c'est de l'état de l'art 

- état de l'art : plus général ce qui est fait et existe sur le sujet, puis ce qu'on a fait, les choix par rapport à ce qui est présenté dans l'état de l'art et les raisons de ces choix. D'où on part, ce qui nous a motivé dans nos choix et où on arrive: résultats. 

- regarder ce qui a été montrer dans le cour
- détecter les canard : très aigu ou très non périodique et 

- limiter frequences à 1500 et mettre une frequences propres d'anche à 2000 
Mais il faudrai avoir intégré dynamique de l'anche 
- carto frequence et amplitude 

changer la longueur du tube pour avoir plus forte et justesse 
